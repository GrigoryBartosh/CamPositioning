\documentclass[12pt]{article}

\usepackage{cmap}
\usepackage[T2A]{fontenc}
\usepackage[utf8]{inputenc}
\usepackage[russian, english]{babel}
\usepackage{graphicx}
\usepackage{amsthm,amsmath,amssymb}
\usepackage[russian, english]{hyperref}
\usepackage{enumerate}
\usepackage{datetime}
\usepackage{listings}

\voffset=-20mm
\textheight=235mm
\hoffset=-25mm
\textwidth=180mm
\headsep=12pt
\footskip=20pt

\newenvironment{MyList}[1][4pt]{
  \begin{enumerate}[1.]
  \setlength{\parskip}{0pt}
  \setlength{\itemsep}{#1}
}{       
  \end{enumerate}
}
\newenvironment{InnerMyList}[1][0pt]{
  \vspace*{-0.5em}
  \begin{enumerate}[a)]
  \setlength{\parskip}{#1}
  \setlength{\itemsep}{0pt}
}{
  \end{enumerate}
}

\begin{document}
	\begin{center}
		{\bf Задача} 
	\end{center}
	Дано:
	\begin{MyList}
		\item
		Модель камеры:
		\begin{InnerMyList}
			\item 
			Разрешение в пикселях: 
			$$Width \times Height$$
			
			\item 
			Угол обзора по вертикали: 
			$$\alpha_{Height}$$
		\end{InnerMyList}
		Известно, что изображение камеры --- это проекция пространства на прямоугольник с соотношением стронами $Width : Height$ через точку, расположенную за прямоугольником на его центральном перпендикуляре --- точку фокуса. 
		
		\item
		$n$ точек:
		\begin{InnerMyList}
			\item 
			Для каждой точки реальные координаты в трехмерном пространстве: 
			$$ X = \begin{pmatrix} 
			x_r \\
			y_r \\
			z_r
			\end{pmatrix} $$
			
			\item 
			Для каждой точки координаты на изображение: 
			$$ P_{pix} = \begin{pmatrix} 
			u_{pix} \\
			v_{pix}
			\end{pmatrix} $$
		\end{InnerMyList}
	\end{MyList}
	По данной информации требуется найти трехмерную матрицу трансформации камеры.
	
	\begin{center}
		{\bf Решение} 
	\end{center}
	Прежде чем решать задачу введем некоторые договоренности
	\begin{InnerMyList}
		\item
		Оси координат расположены так, что:
		$$OZ = OX \times OY$$ 
		
		\item
		Трехмерная матрица трансформации имеет следующий вид:
		$$ T = \left(\begin{array}{@{}c|c@{}}
		R & \begin{matrix} t_x \\ t_y \\ t_z \end{matrix}
		\end{array}\right) $$
		, где $R$ --- трехмерная матрица поворота, а $t = \begin{pmatrix} 
		t_x \\
		t_y \\
		t_z
		\end{pmatrix}$ --- ввектор сдвига.
	\end{InnerMyList}
	
	\newpage
	\begin{MyList}
		\item{Перевод пиксельных координат в гомогенные.}\\
		Воспользуемся информацией о модели камеры и найдем гомогенные координаты точек.
		
		Пусть начало координат находится в точки фокуса, а центр прямоугольника имеет координаты:
		$$ \begin{pmatrix} 
		0 \\
		0 \\
		1
		\end{pmatrix} $$
		Тогда на изображение ось $OX$ будет направлена вправо, а ось $OY$ вниз.
		
		Рассмотрим треугольник образованный множеством точек изображения с координатой $0$ по $OX$ и точкой фокуса. Заметим, что имеет место следущее равенство:
		$$\frac{v_{pix} - Height/2}{Height/2} = \frac{v}{MAX_y}$$
		, где $v$ --- реальная координата по оси $OY$, а $MAX_y$ --- максимальная.\\
		
		Здесь мы переводим $v_{pix}$ из промежутка $[0, Height]$ в промежуток $[-Height/2, Height/2]$, а затем записываем равенство отношений пикселбных координат и реальных.\\
		Отсюда получаем:
		$$v = (v_{pix} - Height/2) \frac{MAX_y}{Height/2}$$
		
		Теперь заметим, что:
		$$MAX_y = \tan(\frac{\alpha_{Height}}{2})$$
		Так как мы принял, что изображение находится на расстоянии $1$ от точки фокуса. А следовательно:
		$$v = (v_{pix} - Height/2) \frac{\tan(\frac{\alpha_{Height}}{2})}{Height/2}$$
		
		Пусть $F$ - фокусное расстояние в пикселях. В таком случае:
		$$\frac{Height/2}{F} = \tan(\frac{\alpha_{Height}}{2}) \Leftrightarrow \frac{1}{F} = \frac{\tan(\frac{\alpha_{Height}}{2})}{Height/2}$$
		Получается, что вычислив $\frac{1}{F}$ один раз, мы можем с легкостью переводить координаты:
		$$v = (v_{pix} - Height/2) \frac{1}{F}$$
		
		По аналогии с этим можно получить и формулу для второй координаты:
		$$u = (u_{pix} - Width/2) \frac{1}{F}$$
		, где $u$ реальная координата по оси $OX$.
		
		Итого мы нашли гомогенные координаты:
		$$P_H := \begin{pmatrix} 
		u \\
		v \\
		1
		\end{pmatrix}$$
		
		\item{Функция минимизации.}\\
		Чего мы хотим? Мы хотим найти матрицу трансформации, соответствующую такому положению камеры, что трехмерные точки при проецирование на изображение совпадали бы с реальными позициями этих точек.
		
		Рассмотрим идеальную ситуацию, когда проекции совпадают с реальными положениями. Это значит, что имеет место равенство:
		$$\lambda P_H = T X_H$$
		, где $\lambda$ --- некий коэффициент, $T$ --- матрица трансформации, а $X_H =  \begin{pmatrix} 
		X \\\hline
		1
		\end{pmatrix}$
		
		А это условие можно записать и через векторное произведение:
		$$P_H \times T X_H = 0$$
		
		Введем кососимметричную матрицу $P$:
		$$P := \begin{pmatrix} 
		0 & -1 & v \\
		1 & 0 & -u \\
		-v & u & 0
		\end{pmatrix}$$
		
		Тогда предыдущее равенство препишется так:
		$$P T X_H = 0$$
		
		Исходя из этого нам нужно подобрать такую матрицу трансформации чтобы сумма значений:
		$$(P T X_H) \cdot (P T X_H) \rightarrow \min$$
		cтремилась к нулю.
		
		\item{Трансформации при малых углах.}\\
		Матрицу поворота можно записать, как произведение матриц поворота вокруг осей:
		$$R = R_z(\omega_z)R_y(\omega_y)R_x(\omega_x) = 
		\begin{pmatrix} 
		\cos{\omega_z} & -\sin{\omega_z} & 0 \\
		\sin{\omega_z} & \cos{\omega_z} & 0 \\
		0 & 0 & 1
		\end{pmatrix}
		\begin{pmatrix} 
		\cos{\omega_y} & 0 & \sin{\omega_y} \\
		0 & 1 & 0 \\
		-\sin{\omega_y} & 0 & \cos{\omega_y}
		\end{pmatrix}
		\begin{pmatrix} 
		1 & 0 & 0 \\
		0 & \cos{\omega_x} & -\sin{\omega_x} \\
		0 & \sin{\omega_x} & \cos{\omega_x}
		\end{pmatrix}$$
		
		Запишем разложение $\sin$ и $\cos$ в ряд до $\mathcal{O}(x^2)$:
		$$\sin{x} = x + \mathcal{O}(x^2), \cos{x} = 1 + \mathcal{O}(x^2)$$
		
		Тогда при достаточно малых углах (до $\approx 20$ градусов) можно линеризовать матрицу поворота:
		$$ R = 
		\begin{pmatrix} 
		1 & -\omega_z & 0 \\
		\omega_z & 1 & 0 \\
		0 & 0 & 1
		\end{pmatrix}
		\begin{pmatrix} 
		1 & 0 & \omega_y \\
		0 & 1 & 0 \\
		-\omega_y & 0 & 1 
		\end{pmatrix}
		\begin{pmatrix} 
		1 & 0 & 0 \\
		0 & 1 & -\omega_x \\
		0 & \omega_x & 1
		\end{pmatrix} =
		\begin{pmatrix} 
		1 & -\omega_z & \omega_y \\
		\omega_z & 1 & -\omega_x \\
		-\omega_y & \omega_x & 1
		\end{pmatrix}$$
		
		Получается, что матрица трансформации будет иметь вид:
		$$ T = \left(\begin{array}{@{}ccc|c@{}}
		1 & -\omega_z & \omega_y & t_x \\
		\omega_z & 1 & -\omega_x & t_y \\
		-\omega_y & \omega_x & 1 & t_z
		\end{array}\right) $$
		
		\newpage
		\item{Нахождение малых углов.}\\
		Во $2$ом пункте мы выяснили, что нужно минимизировать сумму значений:
		$$(P T X_H) \cdot (P T X_H) \rightarrow \min$$
		
		Давайте повнимательнее посмотрим на то, что здесь написано:
		$$T X_H = 
		\left(\begin{array}{@{}ccc|c@{}}
		1 & -\omega_z & \omega_y & t_x \\
		\omega_z & 1 & -\omega_x & t_y \\
		-\omega_y & \omega_x & 1 & t_z
		\end{array}\right)
		\begin{pmatrix} 
		x \\
		y \\
		z \\
		1
		\end{pmatrix} =
		\begin{pmatrix} 
		x - \omega_z y + \omega_y z + t_x \\
		y + \omega_z x - \omega_x z + t_y \\
		z - \omega_y x + \omega_x y + t_z
		\end{pmatrix} +
		\begin{pmatrix} 
		x \\
		y \\
		z
		\end{pmatrix} 
		$$
		
		Пусть:
		$$G = 
		\begin{pmatrix} 
		1 & 0 & 0 &  0 &  z & -y \\
		0 & 1 & 0 & -z &  0 &  x \\
		0 & 0 & 1 &  y & -x &  0 \\
		\end{pmatrix},
		\mu = 
		\begin{pmatrix} 
		t_x \\
		t_y \\
		t_z \\
		\omega_x \\
		\omega_y \\
		\omega_z 
		\end{pmatrix}$$
		
		Используя введенные обозначения можем переписать равенство:
		$$T X_H = G \mu + X$$
		
		Теперь перепишем функцию минимизации из 2го пункта:
		$$(P G \mu + P X) \cdot (P G \mu + P X) \rightarrow \min$$
		
		Для нахождения точки минимума возьмем производную и приравняем ее к $0$ (числовые коэффициенты опущены):
		$$G^T P^T (P G \mu + P X) = 0$$
		
		Получаем простое уравнение:
		$$G^T P^T P G \mu = - G^T P^T P X$$
		
		Такое уравнение нетрудно решить. Для учета всех точек достаточно сложить уравнения:
		$$(\sum G^T P^T P G) \mu = - \sum G^T P^T P X$$
		
		\item{Получение матрицы трансформации для малых углов.}\\
		В предыдущем пункте мы нашли углы. Для восстановления матрицы трансформации по этим углам воспользуемся следующей формулой:
		$$ R = I + \frac{\sin{\theta}}{\theta} \omega_{\times} + \frac{1 - \cos{\theta}}{\theta^2} \omega_{\times}^2 $$
		, где $I$ --- единичная матрица, $\theta = \sqrt{\omega_x^2 + \omega_y^2 + \omega_z^2}$, а $\omega_{\times} = 
		\begin{pmatrix} 
		0 & -\omega_z & \omega_y \\
		\omega_z & 0 & -\omega_x \\
		-\omega_y & \omega_x & 0
		\end{pmatrix}$
		
		\newpage
		\item{Вычисление ответа.}\\
		На данный момент мы имеем алгоритм нахождения матрицы трансформации. Но он не дает точного результата, так как мы линеризуем матрицы поворота. Для нахождения точного результата будем действовать итеративно. Пусть изначально матрица трансформации равна:
		$$T = \left(\begin{array}{@{}ccc|c@{}}
		1 & 0 & 0 & 0 \\
		0 & 1 & 0 & 0 \\
		0 & 0 & 1 & 0
		\end{array}\right)$$
		То есть она никак не трансформирует точки. Далее придерживаемся следующего алгоритма:
		\begin{InnerMyList}
			\item
			Переведем точки в систему координат камеры, соответствующей текущей трансформации:
			$$X' = T X$$
			
			\item 
			Найдем новую матрицу трансформации $T'$, используя в качестве исходных координат точек, координаты $X'$.
			
			\item
			Пересчитаем текущую матрицу трансформации следующим образом:
			$$T = T' T$$
			Для перемножения матриц, к каждой снизу дописывается строка 
			$\begin{pmatrix} 0 & 0 & 0 & 1 \end{pmatrix}$.
			
			\item
			Перейдем в пункт $a)$.
		\end{InnerMyList}
		
		Таким образом, шаг за шагом, мы будем приближаться к верному ответу. Сходится это довольно быстро, так что $20$ итераций хватает слихвой.\\
		
		Мы ставили задачу сопоставить точки на изображение и проекции трехмерных точек. Эту задачу можно сформулировать и по-другому:\\
		Точка на изображение задает луч, исходящий из точки фокуса камеры. Нужно найти такое положение камеры, в котором лучи проходили бы через свои точки в трехмерном пространстве.\\
		Однако, важно понимать следующее: наш метод устроен так, что он работает не с лучами, а с прямыми. А значит на кадр может спроецироваться точка, которая в действительности не видна, которая находиться на дополнение луча до прямой.\\
		В связи с этим может сложиться ситуация, при которой ни одна из точек не видна. Поэтому после нахождения трансформации необходимо проверить, верно ли направлена камера, и если нет, перевернуть ее и вновь посчитать трансформацию, используя трансформацию перевернутой камеры в качестве стартовой.\\
		
		Можно также отметить, что из-за неидеальности камер такая перевернутая трансформация даст большую ошибку, чем трансформация верно установленной камеры. Исходя из этого можно сделать несколько попыток работы алгоритма описанного выше, но использовать не единичную матрицу, а случайную матрицу поворота. После чего просто выбрать трансформацию дающую наименьшую ошибку (что такое ошибка можно вспомнить во втором пункте).\\
		
		Вышесказанное стоит применять, когда матрица трансформации ищется впервые. В случае же, когда мы работаем с потоком виде, стоит использовать матрицу трансформации предыдущего кадра, так как она близка к искомой. Это может значительно ускорить работу и избавить от необходимости лишних проверок.
	\end{MyList}
	
\end{document}